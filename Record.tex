% Options for packages loaded elsewhere
\PassOptionsToPackage{unicode}{hyperref}
\PassOptionsToPackage{hyphens}{url}
\PassOptionsToPackage{dvipsnames,svgnames,x11names}{xcolor}
\documentclass[
  11pt,
]{article}
\usepackage{xcolor}
\usepackage[margin=2.5cm]{geometry}
\usepackage{amsmath,amssymb}
\setcounter{secnumdepth}{5}
\usepackage{iftex}
\ifPDFTeX
  \usepackage[T1]{fontenc}
  \usepackage[utf8]{inputenc}
  \usepackage{textcomp} % provide euro and other symbols
\else % if luatex or xetex
  \usepackage{unicode-math} % this also loads fontspec
  \defaultfontfeatures{Scale=MatchLowercase}
  \defaultfontfeatures[\rmfamily]{Ligatures=TeX,Scale=1}
\fi
\usepackage{lmodern}
\ifPDFTeX\else
  % xetex/luatex font selection
\fi
% Use upquote if available, for straight quotes in verbatim environments
\IfFileExists{upquote.sty}{\usepackage{upquote}}{}
\IfFileExists{microtype.sty}{% use microtype if available
  \usepackage[]{microtype}
  \UseMicrotypeSet[protrusion]{basicmath} % disable protrusion for tt fonts
}{}
\makeatletter
\@ifundefined{KOMAClassName}{% if non-KOMA class
  \IfFileExists{parskip.sty}{%
    \usepackage{parskip}
  }{% else
    \setlength{\parindent}{0pt}
    \setlength{\parskip}{6pt plus 2pt minus 1pt}}
}{% if KOMA class
  \KOMAoptions{parskip=half}}
\makeatother
\usepackage{longtable,booktabs,array}
\usepackage{calc} % for calculating minipage widths
% Correct order of tables after \paragraph or \subparagraph
\usepackage{etoolbox}
\makeatletter
\patchcmd\longtable{\par}{\if@noskipsec\mbox{}\fi\par}{}{}
\makeatother
% Allow footnotes in longtable head/foot
\IfFileExists{footnotehyper.sty}{\usepackage{footnotehyper}}{\usepackage{footnote}}
\makesavenoteenv{longtable}
\usepackage{graphicx}
\makeatletter
\newsavebox\pandoc@box
\newcommand*\pandocbounded[1]{% scales image to fit in text height/width
  \sbox\pandoc@box{#1}%
  \Gscale@div\@tempa{\textheight}{\dimexpr\ht\pandoc@box+\dp\pandoc@box\relax}%
  \Gscale@div\@tempb{\linewidth}{\wd\pandoc@box}%
  \ifdim\@tempb\p@<\@tempa\p@\let\@tempa\@tempb\fi% select the smaller of both
  \ifdim\@tempa\p@<\p@\scalebox{\@tempa}{\usebox\pandoc@box}%
  \else\usebox{\pandoc@box}%
  \fi%
}
% Set default figure placement to htbp
\def\fps@figure{htbp}
\makeatother
\setlength{\emergencystretch}{3em} % prevent overfull lines
\providecommand{\tightlist}{%
  \setlength{\itemsep}{0pt}\setlength{\parskip}{0pt}}
\usepackage{float}
\usepackage{booktabs}
\usepackage{longtable}
\usepackage{fancyhdr}
\usepackage{pdfpages}
\setlength{\headheight}{14pt}
\pagestyle{fancy}
\fancyhead[R]{\thepage}
\fancyfoot[C]{}
\floatplacement{figure}{H}
\usepackage{booktabs}
\usepackage{longtable}
\usepackage{array}
\usepackage{multirow}
\usepackage{wrapfig}
\usepackage{float}
\usepackage{colortbl}
\usepackage{pdflscape}
\usepackage{tabu}
\usepackage{threeparttable}
\usepackage{threeparttablex}
\usepackage[normalem]{ulem}
\usepackage{makecell}
\usepackage{xcolor}
\usepackage{bookmark}
\IfFileExists{xurl.sty}{\usepackage{xurl}}{} % add URL line breaks if available
\urlstyle{same}
\hypersetup{
  colorlinks=true,
  linkcolor={black},
  filecolor={Maroon},
  citecolor={Blue},
  urlcolor={black},
  pdfcreator={LaTeX via pandoc}}

\author{}
\date{\vspace{-2.5em}}

\begin{document}

\includepdf{medias/page_de_garde.pdf}
\newpage

\pagenumbering{gobble}
\thispagestyle{empty}
\vspace*{\fill}
\begin{flushleft}
\fontsize{11}{13}\selectfont
\textit{Le Sèmè City Open Park est un lieu de libération. Un espace où l'on désapprend la peur de créer, où l'éducation devient action, et où chaque idée, même fragile, a le droit d'exister. C'est un terrain d'expérimentation, de résilience et de partage, où l'on fait plus avec moins, ensemble. Un endroit où l'Afrique ne se rêve pas en retard, mais se construit, ici et maintenant, par celles et ceux qui osent essayer.}
\end{flushleft}
\newpage

\pagenumbering{arabic}
\setcounter{page}{1}

\newpage
\pagenumbering{roman}
\setcounter{page}{1}

\tableofcontents
\newpage

\pagenumbering{arabic}
\setcounter{page}{1}

\section{Résumé de l'étude}\label{ruxe9sumuxe9-de-luxe9tude}

\subsection{Vers une qualification des usagers du Sèmè City Open
Park}\label{vers-une-qualification-des-usagers-du-suxe8muxe8-city-open-park}

\subsection{Que faire dès aujourd'hui
?}\label{que-faire-duxe8s-aujourdhui}

\newpage

\section{Introduction}\label{introduction}

Un tiers lieu comme le Sèmè City Open Park ne vit pas uniquement de ses
infrastructures, de ses machines ou de ses programmes. Il vit avant tout
de celles et ceux qui le traversent, qui s'y arrêtent, qui partent et
qui y reviennent\ldots{} Depuis son ouverture au public en avril 2022,
le Sèmè City Open Park a vu défiler une grande diversité de profils.
Certains viennent d'abord par curiosité, d'autres par nécessité. Mais
après quelques jours à vivre cette expérience qui n'existe nulle part
ailleurs en ce monde, quelques-uns finissent par rester. Cette pluralité
est une force, mais elle rend difficile toute lecture intuitive des
dynamiques de fréquentation. En effet, ce que l'on croit percevoir à
l'œil nu ne reflète pas toujours la réalité des usages.

Améliorer la fréquentation d'un tel espace ne consiste donc pas
simplement à augmenter des chiffres. C'est une démarche plus profonde,
qui interroge le rapport entre un lieu et ses usagers, entre une
promesse et son appropriation réelle. Dans ce contexte, la donnée
devient une mémoire silencieuse. Elle conserve les traces des passages,
des habitudes, des absences. Encore faut-il savoir l'écouter, la
nettoyer, la structurer et l'interpréter. C'est précisément l'ambition
de ce travail, articulé autour de la question suivante : comment
améliorer la fréquentation du Sèmè City Open Park à partir de l'analyse
des données d'usage disponibles ?

Pour y répondre, cette étude mobilise une démarche rigoureuse d'analyse
de données, allant du nettoyage avancé des bases existantes à leur
enrichissement par des variables temporelles et comportementales.
L'objectif n'est pas uniquement de regarder en arrière, mais de faire
émerger des signaux utiles pour l'action, capables d'éclairer les
décisions stratégiques à venir.

\section{Méthodologie de l'étude}\label{muxe9thodologie-de-luxe9tude}

Comprendre la fréquentation du Sèmè City Open Park ne peut heuresement
se faire à partir d'une seule photographie figée. Il a fallu retrousser
ses manches, collecter de la matière brute, très imparfaite et
fragmentée, que constituent les données. Nos choix méthodologiques ont
été fait pour respecter la réalité du lieu telle qu'elle est vécue, tout
en lui donnant une forme analytique exploitable.

\subsection{Sources de données
utilisées}\label{sources-de-donnuxe9es-utilisuxe9es}

Cette étude repose sur 03 sources de données principales. La première,
SmartSCOP, un logiciel interne au SCOP qui nous aide à retracer les flux
d'entrée sortie du SCOP. C'est le régistre numérique qu'utilisent les
usagers lorsqu'ils visitent le Sèmè City Open Park. De cet outil, nous
avons extrait au format excel, deux bases essentielles :

\begin{itemize}
\item
  data\_presences\_raw, qui contient contient 14809 observations et 10
  variables. Ces données renseignent sur les passages mesurés au sein du
  parc ;
\item
  data\_usagers\_raw, qui contient contient 3553 observations et 14
  variables. Elle renseigne sur les usagers qui fréquentent le SCOP ;
\end{itemize}

Prises isolément, ces sources offrent une vision partielle. Croisées et
harmonisées, elles deviennent une mémoire collective du lieu.

\subsection{Nettoyage et transformation des
données}\label{nettoyage-et-transformation-des-donnuxe9es}

\subsubsection{Nettoyage des entités
HTML}\label{nettoyage-des-entituxe9s-html}

Les enrégistrement sur les espaces visités par les usagers dans la base
de données de présence, de même que les motifs de visite présentaient
des traces techniques, notamment des entités HTML parasites (comme
\texttt{\&amp;} pour -) issues des outils de saisie. Certains caractères
spéciaux comme les accents, les tierts, apostrophes, étaient donc
camouflés derrière ces entités parasites. Nous avons donc créé une
fonction qui détectait différents types d'entités et les remplaçaient
par le caractère correspondant en encodage UTF-8.

Leur suppression a constitué une étape essentielle pour restaurer la
lisibilité des informations et éviter toute distorsion dans les analyses
ultérieures. Ce travail minutieux a permis de fiabiliser plusieurs
milliers d'enregistrements et de réduire significativement le bruit dans
les variables textuelles dans les bases de données brutes.

\subsubsection{Harmonisation des variables
qualitatives}\label{harmonisation-des-variables-qualitatives}

Sur des variables comme l'université d'appartenance, le profil ou même
la ville de résidence dans la base de données des usagers, une même
information pouvait être orthographiée différemment, abrégée ou saisie
de manière approximative. Pour répondre à cette hétérogénéité des
données, des techniques de distance de chaînes de Jaro-Winkler et de
vérification d'acronymes ont été mobilisées.

Pour rappel, la distance de chaîne (ou distance d'édition) mesure la
similarité entre deux chaînes de caractères en comptant le nombre
minimal d'opérations nécessaires pour transformer l'une en l'autre. La
méthode de Jaro-Winkler employée ici donne plus de poids aux
correspondances au début de la chaîne (utile pour les noms), qui était
l'idéale pour les données en présence (nom d'universités, de
compétences, de profils). C'est un outil puissant pour gérer les erreurs
de saisie, les variations d'orthographe ou les approximations dans les
données textuelles.

Le tableau ci-dessous présente les résultats de l'harmonisation des
variables textuelles de la base \texttt{data\_usagers}. Valeurs
indiquent le nombre de valeurs uniques détectées avant et après
harmonisation.

\begin{longtable}[]{@{}
  >{\raggedright\arraybackslash}p{(\linewidth - 8\tabcolsep) * \real{0.2286}}
  >{\raggedleft\arraybackslash}p{(\linewidth - 8\tabcolsep) * \real{0.2000}}
  >{\raggedleft\arraybackslash}p{(\linewidth - 8\tabcolsep) * \real{0.2000}}
  >{\raggedleft\arraybackslash}p{(\linewidth - 8\tabcolsep) * \real{0.1429}}
  >{\raggedleft\arraybackslash}p{(\linewidth - 8\tabcolsep) * \real{0.2286}}@{}}
\caption{Résultats de l'harmonisation des variables
textuelles}\tabularnewline
\toprule\noalign{}
\begin{minipage}[b]{\linewidth}\raggedright
Variable
\end{minipage} & \begin{minipage}[b]{\linewidth}\raggedleft
Valeurs avant
\end{minipage} & \begin{minipage}[b]{\linewidth}\raggedleft
Valeurs après
\end{minipage} & \begin{minipage}[b]{\linewidth}\raggedleft
Réduction
\end{minipage} & \begin{minipage}[b]{\linewidth}\raggedleft
Equivalence (\%)
\end{minipage} \\
\midrule\noalign{}
\endfirsthead
\toprule\noalign{}
\begin{minipage}[b]{\linewidth}\raggedright
Variable
\end{minipage} & \begin{minipage}[b]{\linewidth}\raggedleft
Valeurs avant
\end{minipage} & \begin{minipage}[b]{\linewidth}\raggedleft
Valeurs après
\end{minipage} & \begin{minipage}[b]{\linewidth}\raggedleft
Réduction
\end{minipage} & \begin{minipage}[b]{\linewidth}\raggedleft
Equivalence (\%)
\end{minipage} \\
\midrule\noalign{}
\endhead
\bottomrule\noalign{}
\endlastfoot
Université & 287 & 181 & 106 & 36.9 \\
Ville & 141 & 108 & 33 & 23.4 \\
Profil & 515 & 342 & 173 & 33.6 \\
Filière d'étude & 391 & 252 & 139 & 35.5 \\
\end{longtable}

L'harmonisation a donc permis de réduire significativement le nombre de
variations pour chaque variable, avec une réduction moyenne de 32.4\%
des valeurs uniques. La variable la plus impactée est l'université avec
une réduction de 36.9\%, tandis que la ville présente le taux le plus
faible à 23.4\%.

\subsubsection{Clé d'identification des
usagers}\label{cluxe9-didentification-des-usagers}

Après observations de la base de donnée des présences, le numéro de
téléphone a été retenu comme clé principale d'identification des
usagers, car il constitue l'information la plus stable et la plus
fréquemment renseignée. Ce choix, ancré dans la réalité opérationnelle
du terrain, a permis de relier efficacement les informations de
présences aux usagers uniques qui les génèrent.

\subsubsection{Panelisation des tables}\label{panelisation-des-tables}

Afin de capter la dynamique temporelle de la fréquentation, les données
ont été panelisées selon deux axes complémentaires :

\begin{itemize}
\item
  un panel journalier, permettant d'observer l'évolution de la
  fréquentation dans le temps ;
\item
  un panel par usager, permettant de reconstituer des comportements
  individuels propres à chacun.
\end{itemize}

Cette double lecture a révélé que la fréquentation est loin d'être
linéaire. Certains jours concentrent une activité très élevée, tandis
que d'autres restent en retrait. De la même manière, quelques usagers
concentrent une part importante des passages, illustrant un phénomène
classique de type longue traîne que nous détaillerons dans la
cartographie de la fréquentation du SCOP.

\subsubsection{Feature engineering}\label{feature-engineering}

Pour dépasser une lecture brute des présences, un travail
d'enrichissement a été mené à travers la création de nouvelles variables
:

\begin{itemize}
\tightlist
\item
  des variables temporelles (jour de la semaine, effets calendaires) ;
\item
  des indicateurs de récurrence ;
\item
  des moyennes flottantes permettant de lisser les variations
  quotidiennes.
\end{itemize}

Ces features ont joué un rôle clé dans l'analyse. Par exemple, certaines
variables temporelles expliquent à elles seules plus d'un tiers de la
variabilité observée dans la fréquentation, confirmant le rôle central
des rythmes hebdomadaires et organisationnels.

\subsection{Construction de la base de données
d'analyse}\label{construction-de-la-base-de-donnuxe9es-danalyse}

Après un nettoyage approfondi et une harmonisation des variables
textuelles par détection d'acronymes et calcul de distance de chaînes,
ces deux bases ont été fusionnées par numéro de téléphone avec un taux
de correspondance de 100\% (3540 téléphones correspondants). Cette
jointure a permis de créer deux bases de travail complémentaires.

La première, \texttt{data\_frequentation}, contient 14809 observations
correspondant aux visites individuelles enrichies des informations de
profil (âge, université, filière, etc.) et des variables calculées
(durée de visite, ancienneté journalière). Cette base présente un taux
de complétude élevé, avec seulement 0,3\% de valeurs manquantes pour les
variables de flux et entre 22,6\% et 57,6\% pour les variables de profil
optionnelles pour les usagers.

La seconde base, \texttt{data\_usagers\_comportement}, agrège ces
visites au niveau usager et contient 3540 individus uniques caractérisés
par 30 variables comportementales (nombre de visites, durées moyennes,
espaces fréquentés, régularité).

Ces deux bases complémentaires permettent respectivement des analyses au
niveau des flux de fréquentation et au niveau des comportements
individuels des usagers.

\subsection{Traitement des valeurs manquantes
(NA)}\label{traitement-des-valeurs-manquantes-na}

Les valeurs manquantes dans les deux bases de données créées présentent
des profils distincts liés à leur nature respective. Dans
\texttt{data\_frequentation}, les données manquantes sont principalement
concentrées sur les variables de profil optionnelles : 57.6\% pour la
filière d'étude, 55.8\% pour l'université et 22.6\% pour le profil. Ces
valeurs manquantes reflètent le caractère facultatif de ces informations
lors de l'inscription des usagers et ne compromettent pas les analyses
de flux temporel ou spatial. En revanche, les variables critiques de
fréquentation (dates d'arrivée, espaces visités) ne présentent aucune
valeur manquante, garantissant l'intégrité des analyses
comportementales. Seule la variable \texttt{departure\_time} affiche un
taux de 0,3\% de valeurs manquantes, correspondant à des sessions où
l'heure de départ n'a pas été enregistrée.

Dans \texttt{data\_usagers\_comportement}, les valeurs manquantes
suivent la même logique puisque cette base agrège les informations au
niveau usager. Les variables comportementales calculées (nombre de
visites, durées moyennes, régularité) ne contiennent aucune valeur
manquante car elles sont dérivées directement des enregistrements de
présence. Les variables sociodémographiques (âge, sexe, handicap)
présentent également un taux de complétude élevé, car elles constituent
des informations de base collectées systématiquement.

Le choix a été fait de \textbf{conserver toutes les observations} avec
leurs valeurs manquantes plutôt que de supprimer des lignes, afin de
maximiser la taille de l'échantillon pour les analyses. Les valeurs
manquantes seront traitées de manière spécifique selon le type d'analyse
: exclusion ponctuelle pour les analyses nécessitant la variable
concernée, imputation par la modalité la plus fréquente pour certaines
modélisations, ou création d'une catégorie ``Non renseigné'' pour les
analyses descriptives conservant l'information sur le non-renseignement
lui-même.

\subsection{Packages et librairies
utilisées}\label{packages-et-librairies-utilisuxe9es}

L'analyse de la fréquentation du Sèmè City Open Park s'appuie sur un
écosystème de 56 packages R spécialisés, couvrant l'ensemble de la
chaîne analytique. La manipulation des données repose principalement sur
le \textbf{tidyverse} (dplyr, tidyr, stringr) complété par
\textbf{data.table} pour les opérations volumineuses, et
\textbf{janitor} pour le nettoyage initial. Le traitement des dates et
heures utilise \textbf{lubridate} et \textbf{hms}, essentiels pour
l'analyse temporelle de la fréquentation. L'harmonisation des variables
textuelles mobilise \textbf{stringdist} pour le calcul des distances de
chaînes et \textbf{fuzzyjoin} pour les jointures approximatives,
permettant de réduire les variations orthographiques de 23\% à 37\%
selon les variables.

La phase exploratoire combine \textbf{summarytools} pour les
statistiques descriptives, \textbf{naniar} et \textbf{VIM} pour
l'analyse des valeurs manquantes, ainsi que \textbf{psych} pour les
corrélations et les statistiques multivariées. La visualisation
s'articule autour de \textbf{ggplot2} enrichi de plusieurs extensions :
\textbf{patchwork} pour l'assemblage de graphiques, \textbf{ggridges}
pour les distributions temporelles, \textbf{viridis} pour les palettes
de couleurs accessibles, et \textbf{plotly} pour l'interactivité. Les
analyses spatiales utilisent \textbf{leaflet} pour la cartographie
interactive et \textbf{rnaturalearth} pour les fonds de carte.

Pour la modélisation prédictive et le machine learning, l'environnement
intègre \textbf{caret} comme framework unifié, \textbf{randomForest} et
\textbf{xgboost} pour les méthodes ensemblistes, \textbf{glmnet} pour la
régularisation, et \textbf{pROC} pour l'évaluation des performances.
L'analyse multivariée repose sur \textbf{FactoMineR} et
\textbf{factoextra} pour les ACP et classifications, tandis que les
prévisions temporelles mobilisent \textbf{forecast} et \textbf{prophet}.
L'interprétabilité des modèles est assurée par \textbf{pdp} (Partial
Dependence Plots), \textbf{iml} et \textbf{DALEX} pour comprendre les
contributions des variables. Enfin, la gestion rigoureuse des conflits
de namespace entre packages (notamment entre plyr/dplyr et
base/lubridate) est orchestrée par \textbf{conflicted}, garantissant la
reproductibilité des analyses et évitant les comportements inattendus
dans le pipeline de traitement.

\section{Présentation du Sèmè City Open
Park}\label{pruxe9sentation-du-suxe8muxe8-city-open-park}

(Commencer un premier paragraphe)

La fréquentation du Sèmè City Open Park n'est pas un problème à
résoudre, mais une dynamique à lire, à accompagner et à amplifier.

\section{Orientation stratégique de
l'analyse}\label{orientation-stratuxe9gique-de-lanalyse}

Toute analyse de données implique des choix. Choisir, c'est renoncer.
Dans le cadre de cette étude, ces choix n'ont pas été dictés par la
facilité, mais par une volonté claire : produire une analyse utile,
actionnable et alignée avec les enjeux réels du Sèmè City Open Park.

\subsection{Durée considérée pour
l'analyse}\label{duruxe9e-considuxe9ruxe9e-pour-lanalyse}

Bien que le SCOP opère des activités depuis avril 2022, l'analyse s'est
concentrée sur une période de 245, du 2025-04-09 au 2025-12-09. Cela
correspond à la période d'exploitation de l'outil smartscop, déployé
plus tôt cette année.

Ce n'est qu'une goute d'eau dans l'histoire du SCOP, et pas de la
meilleure période, je peux vous l'affirmer. Cependant, cette durée
permet quand même d'observer :

\begin{itemize}
\tightlist
\item
  des cycles hebdomadaires nets ;
\item
  des variations saisonnières ;
\item
  et des phénomènes de récurrence chez certains usagers.
\end{itemize}

Pourquoi ne pas êtr remonté plus loin ? En réalité, nous dispositions en
version papier de plusieurs régistres qui renseignent les flux d'usagers
depuis l'ouverture, et d'archives de listes de présence aux activités
clés menées sur le site depuis 2022.

Une démarche d'extraction des données a été initiée au moment de mon
stage, donnant naissance à une base de données d'activité, que nous
avons finalement dû écarter de cette étude.

\subsection{Suppression de la base de données
d'activité}\label{suppression-de-la-base-de-donnuxe9es-dactivituxe9}

Ce choix peut sembler contre-intuitif. Pourtant, il répond à une logique
claire. Premièrement, à peine 38 \% des usagers enrégistrés auraient
réellement participé aux activités sur cette période. Cela réflète une
sous-utilisation de l'outil smartscop, pourtant dédié à la gestion du
flux d'usagers. Ce qui aurait généré une perte énorme d'informations en
cas de jointures.

De plus, l'objectif de l'étude n'était pas d'évaluer la performance
intrinsèque des activités proposées, mais de comprendre les dynamiques
globales de fréquentation indépendamment des formats spécifiques.

Intégrer trop tôt la dimension ``activité'' aurait risqué de brouiller
la lecture, en confondant :

\begin{itemize}
\tightlist
\item
  l'effet du lieu,
\item
  l'effet de la programmation,
\item
  et l'effet des comportements individuels.
\end{itemize}

En se concentrant d'abord sur les présences et les comportements
d'usagers, l'analyse a pu mettre en évidence des régularités robustes.
Ces résultats constituent une base saine sur laquelle des analyses
futures, plus ciblées sur les activités, pourront s'appuyer une fois la
reconstitution des données complétées.

\newpage

\section{Cartographie de la fréquentation du Sèmè City Open
Park}\label{cartographie-de-la-fruxe9quentation-du-suxe8muxe8-city-open-park}

Le Sèmè City Open Park est ouvert au public, du lundi au Samedi, de 09 h
à 22 h inclus. Cette année, nous avons enrégistré 14 809 visites,
réalisées par 3 540 usagers uniques, sur 245 jours d'observation, dont
209 jours effectivement ouverts, soit un taux d'ouverture de 85 \%.

\subsection{Ampleur et structure de la
fréquentation}\label{ampleur-et-structure-de-la-fruxe9quentation}

\begin{center}\includegraphics[width=0.9\linewidth]{outputs/figures/01_frequentation_journaliere} \end{center}

Le moins que l'on puisse dire, c'est que le rythme de fréquentation au
Sèmè City Open Park est très dynamique. Les visites quotidiennes
oscillent entre \texttt{\{r\ min(data\_daily\$nb\_visites)\ \}} (les
jours de fermetture comme le dimanche) et
\texttt{\{r\ max(data\_daily\$nb\_visites)\ \}} visites par jour, avec
une moyenne générale à environ
\texttt{\{r\ round(mean(data\_daily\$nb\_visites),\ 0)\ \}} visites par
jour.

En prenant un peu de recul, on remarque très vite que, mis bout à bout,
ces flux quotidiens génèrent quand même un traffic mensuel assez
important, entre 1352 en Aôut contre 2252 le mois précédent.

\begin{center}\includegraphics[width=0.9\linewidth]{outputs/figures/02_frequentation_mensuelle} \end{center}

Ces chiffres traduisent une activité soutenue, mais surtout une
fréquentation très inégalement répartie entre les différents individus.

\begin{center}\includegraphics[width=0.9\linewidth]{outputs/figures/17_distribution_nb_visites} \end{center}

En effet, l'analyse révèle une structure fortement concentrée :

\begin{itemize}
\tightlist
\item
  55,9 \% des usagers enrégistrés ne sont venus qu'une seule fois ;
\item
  7,9 \% des usagers génèrent à eux seuls 55,4 \% de l'ensemble des
  visites.
\end{itemize}

Cette dissymétrie indique que la fréquentation du lieu repose largement
sur un noyau réduit d'usagers très engagés, autour duquel gravite une
majorité de visiteurs occasionnels.

\subsection{Rythmes, cycles et
contrastes}\label{rythmes-cycles-et-contrastes}

\subsubsection{Journées les plus
fréquentées}\label{journuxe9es-les-plus-fruxe9quentuxe9es}

La fréquentation du Sèmè City Open Park n'est pas constante. Elle suit
des cycles nets, observables à l'échelle mensuelle, hebdomadaire et
journalière. Les analyses temporelles montrent que la fréquentation est
majoritairement concentrée en semaine, avec 83,6 \% des visites, contre
16,4 \% le week-end.

\begin{center}\includegraphics[width=0.9\linewidth]{outputs/figures/03_frequentation_jour_semaine} \end{center}

Le vendredi apparaît comme le jour le plus actif, avec 2894 visites
cumulées, tandis que le dimanche est de loin le jour le moins fréquenté,
avec seulement 95 visites sur l'ensemble de la période. L'écart entre
ces deux journées atteint 132 \%, illustrant un potentiel d'optimisation
considérable.

\subsubsection{Horaires et durées des
visites}\label{horaires-et-duruxe9es-des-visites}

L'analyse des heures d'arrivée montre une fréquentation principalement
concentrée entre 9h et 18h, avec un pic des arrivées marqué autour de
10h et une heure moyenne d'arrivée située à 14h.

\begin{center}\includegraphics[width=0.9\linewidth]{outputs/figures/04_distribution_heures_arrivee} \end{center}

Les durées de visite sont particulièrement révélatrices du type d'usage
du lieu :

\begin{itemize}
\tightlist
\item
  la durée médiane est de 218 minutes (environ 3h40) ;
\item
  la durée moyenne atteint 282 minutes (4h42) ;
\item
  31,7 \% des visites dépassent 6 heures de temps ;
\item
  seulement 18,1 \% des visites durent moins de 60 minutes.
\end{itemize}

\begin{center}\includegraphics[width=0.9\linewidth]{outputs/figures/08_categories_duree} \end{center}

Ces chiffres montrent que le Sèmè City Open Park n'est pas un lieu de
passage rapide, mais un lieu d'ancrage, où les usagers s'installent dans
le temps.

\subsection{Utilisation de l'espace : concentrations et
déséquilibres}\label{utilisation-de-lespace-concentrations-et-duxe9suxe9quilibres}

Le Sèmè City Open Park compte 29 espaces actifs, mais leur fréquentation
est loin d'être homogène. Les résultats montrent une forte concentration
sur très peu d'espaces :

\begin{itemize}
\tightlist
\item
  les 3 espaces les plus fréquentés concentrent 51,8 \% des visites ;
\item
  le Rooftop, à lui seul, totalise 3622 visites, ce qui en fait l'espace
  le plus attractif du site.
\end{itemize}

\begin{center}\includegraphics[width=0.9\linewidth]{outputs/figures/11_top_espaces} \end{center}

À l'inverse, 11 espaces sont identifiés comme sous-utilisés, avec moins
de 50 visites chacun, représentant au total seulement 0,6 \% de la
fréquentation globale.

Cette répartition suggère que l'amélioration de la fréquentation ne
passe pas uniquement par l'acquisition de nouveaux usagers, mais aussi
par une meilleure répartition des flux internes.

\subsection{Profil démographique des
usagers}\label{profil-duxe9mographique-des-usagers}

\subsubsection{Le sexe masculin domine sur le
parc}\label{le-sexe-masculin-domine-sur-le-parc}

Le parc présente une forte disparité de genre avec une prédominance
masculine marquée. Les hommes représentent 75,1\% des usagers contre
24,9\% pour les femmes, soit un ratio de 3:1. Cette asymétrie pose la
question de l'attractivité différenciée du parc selon le genre et
suggère des axes d'amélioration pour diversifier le public.

\subsubsection{L'utilisateur moyen a 25
ans}\label{lutilisateur-moyen-a-25-ans}

La population du parc est majoritairement jeune avec un âge médian de 23
ans et une moyenne de 24 ans. La distribution révèle une forte
concentration dans la tranche 20-24 ans qui représente à elle seule
50,2\% des usagers.

\begin{center}\includegraphics[width=0.9\linewidth]{outputs/figures/09_pyramide_ages} \end{center}

Cette caractéristique s'explique par la proximité d'établissements
universitaires, notamment Epitech, Africa DEsign School et l'Université
d'Abomey-Calavi (UAC), qui constituent le principal réservoir d'usagers.

\subsubsection{Il réside à Cotonou et ses
environs}\label{il-ruxe9side-uxe0-cotonou-et-ses-environs}

Les usagers du Sèmè City Open Park proviennent de 108 villes
différentes, mais cette diversité est très concentrée :

\begin{itemize}
\tightlist
\item
  les 5 premières villes représentent 93,6 \% des visites,
\item
  la ville de Cotonou concentre à elle seule 55,7 \% des visites.
\end{itemize}

\begin{center}\includegraphics[width=0.9\linewidth]{outputs/figures/13_top_villes} \end{center}

L'indice de concentration géographique (HHI = 0,4284) confirme un
ancrage territorial très marqué, avec une faible diffusion géographique
actuelle.

\subsubsection{Il est étudiant en informatique et technologies
numériques}\label{il-est-uxe9tudiant-en-informatique-et-technologies-numuxe9riques}

287 établissements différents sont recensés, témoignant d'une diversité
remarquable. Cependant, le taux de renseignement de 44,2\% indique
qu'une part importante des usagers ne déclare pas leur affiliation, ce
qui peut biaiser l'analyse.

La distribution des domaines d'étude reflète la vocation technologique
et innovante du parc. Avec 391 domaines différents recensés (réduits à
252 après harmonisation), la diversité disciplinaire est remarquable,
bien que le taux de renseignement de 42,4\% limite la portée de cette
analyse.

\subsubsection{Ses motivations sont très
personnelles}\label{ses-motivations-sont-truxe8s-personnelles}

Le travail personnel est le motif le plus utilisé par les usagers du
Sèmè City Open Park, qui bénéficient sans doute de l'expérience SCOP :
un kit de démarrage dans le parc, où ils peuvent accéder à un espace de
travail libre et confortable, de l'électricité et une connexion
internet.

\subsection{Un comportement
hétérogène}\label{un-comportement-huxe9tuxe9roguxe8ne}

\subsubsection{Segmentation par fréquence de
visite}\label{segmentation-par-fruxe9quence-de-visite}

La segmentation comportementale des usagers permet d'identifier quatre
profils principaux de visiteurs :

\begin{itemize}
\tightlist
\item
  Les Occasionnels (1 visite) : 1978 usagers (55,9 \%)
\item
  Les Explorateurs (2--5 visites) : 1050 usagers (29,7 \%)
\item
  Les Réguliers (6--10 visites) : 233 usagers (6,6 \%)
\item
  Les Fidèles (11+ visites) : 279 usagers (7,9 \%)
\end{itemize}

Malgré leur faible proportion, les usagers fidèles génèrent plus de la
moitié des visites totales, avec une durée moyenne par visite de 290
minutes et un score d'engagement maximal.

\subsubsection{Ancienneté et
fidélisation}\label{anciennetuxe9-et-fiduxe9lisation}

L'analyse de l'ancienneté montre une fragilité importante de la
rétention :

\begin{itemize}
\tightlist
\item
  56,7 \% des usagers n'ont qu'un jour d'ancienneté (une seule visite) ;
\item
  73,4 \% des usagers ont une ancienneté inférieure à 30 jours ;
\item
  seuls 16,3 \% dépassent 90 jours d'ancienneté.
\end{itemize}

Par ailleurs, 24,5 \% des visites sont effectuées le jour même de
l'inscription, confirmant le caractère critique et déterminant de la
première expérience pour la suite.

\subsubsection{Préférences au niveau des
espaces}\label{pruxe9fuxe9rences-au-niveau-des-espaces}

L'analyse des comportements individuels révèle une faible mobilité
inter-espaces :

\begin{itemize}
\tightlist
\item
  70,3\% des usagers fréquentent un seul espace (mono-espace) ;
\item
  25,4\% utilisent 2 à 3 espaces ;
\item
  4,3\% diversifient leur usage sur 4 espaces ou plus.
\end{itemize}

Cette concentration pose question sur la découvrabilité et la
polyvalence de l'offre. Les usagers mono-espace représentent un
potentiel de diversification considérable : 2 489 usagers pourraient
être incités à explorer d'autres espaces.

\subsection{En résumé}\label{en-ruxe9sumuxe9}

La cartographie de la fréquentation met en évidence quatre enseignements
structurants :

\begin{itemize}
\tightlist
\item
  La fréquentation est massivement concentrée sur un noyau réduit
  d'usagers engagés.
\item
  Les rythmes temporels jouent un rôle déterminant, avec des écarts très
  forts entre jours et horaires.
\item
  Les espaces ne contribuent pas de manière équivalente à l'attractivité
  du lieu.
\item
  La première visite constitue un moment critique dans la trajectoire
  d'engagement.
\end{itemize}

Ces constats structurent directement les hypothèses analysées dans la
section suivante et fondent les préconisations stratégiques formulées
ultérieurement.

\section{Les milles et un facteurs d'influence de la fréquentation du
parc}\label{les-milles-et-un-facteurs-dinfluence-de-la-fruxe9quentation-du-parc}

Si la fréquentation du Sèmè City Open Park peut, à première vue, sembler
relever d'une alchimie complexe, l'analyse des données révèle au
contraire une réalité plus structurée. Derrière la diversité apparente
des usages se cachent quelques facteurs dominants, capables d'expliquer
à la fois le nombre de visites, leur durée, et la capacité du lieu à
fidéliser ses usagers.

\begin{center}\includegraphics[width=0.9\linewidth]{outputs/figures/pdp_grid_top4} \end{center}

L'analyse de ces facteurs d'influence révèlera d'ailleurs que la
fréquentation du parc est avant tout un phénomène endogène : elle dépend
plus de ce qui se passe à l'intérieur du lieu que de facteurs externes.

\newpage

\section{Discussion : Hypothèses et
Résultats}\label{discussion-hypothuxe8ses-et-ruxe9sultats}

\newpage

\section{Préconisations
Stratégiques}\label{pruxe9conisations-stratuxe9giques}

\newpage

\section{Limites de l'Étude}\label{limites-de-luxe9tude}

\newpage

\section{Conclusion et Synthèse}\label{conclusion-et-synthuxe8se}

\section{Sources}\label{sources}

\newpage

\end{document}
